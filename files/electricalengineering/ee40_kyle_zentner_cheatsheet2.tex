\documentclass[9pt,landscape]{article}
\usepackage{multicol}
\usepackage{calc}
\usepackage{ifthen}
\usepackage[landscape]{geometry}
\usepackage{amsmath,amsthm,amsfonts,amssymb}
\usepackage{framed,tikz,setspace,fancyhdr,enumerate}
\usepackage{color,graphicx,overpic}
\usepackage{hyperref}
\usepackage{vwcol}

\def\Name{Kyle Zentner}
\def\Session{Spring 2014}
\def\Title{EE40 Cheatsheet}

\pagestyle{fancy}
\geometry{letterpaper, margin=1in}

\setstretch{0.1}

\newenvironment{itemize*}%
  {\begin{itemize}%
    \setlength{\itemsep}{1pt}%
    \setlength{\parskip}{1pt}}%
  {\end{itemize}}

\author{\Name}
\lhead{\Title}

\def \ind {\perp\!\!\!\perp}
\def \cupdot {\mathbin{\mathaccent\cdot\cup}}

\pdfinfo{
  /Title (\Title)
  /Creator (TeX)
  /Producer (pdfTeX 1.40.0)
  /Author (\Name)
  /Subject (\Title)
  /Keywords ()}

% This sets page margins to .5 inch if using letter paper, and to 1cm
% if using A4 paper. (This probably isn't strictly necessary.)
% If using another size paper, use default 1cm margins.
\ifthenelse{\lengthtest { \paperwidth = 11in}}
    { \geometry{top=.5in,left=.5in,right=.5in,bottom=.5in} }
    {\ifthenelse{ \lengthtest{ \paperwidth = 297mm}}
        {\geometry{top=1cm,left=1cm,right=1cm,bottom=1cm} }
        {\geometry{top=1cm,left=1cm,right=1cm,bottom=1cm} }
    }

% Turn off header and footer
\pagestyle{empty}

% Redefine section commands to use less space
\makeatletter
\renewcommand{\section}{\@startsection{section}{1}{0mm}%
                                {-1ex plus -.5ex minus -.2ex}%
                                {0.5ex plus .2ex}%x
                                {\normalfont\large\bfseries}}
\renewcommand{\subsection}{\@startsection{subsection}{2}{0mm}%
                                {-1explus -.5ex minus -.2ex}%
                                {0.5ex plus .2ex}%
                                {\normalfont\normalsize\bfseries}}
\renewcommand{\subsubsection}{\@startsection{subsubsection}{3}{0mm}%
                                {-1ex plus -.5ex minus -.2ex}%
                                {1ex plus .2ex}%
                                {\normalfont\small\bfseries}}
\makeatother

% Define BibTeX command
\def\BibTeX{{\rm B\kern-.05em{\sc i\kern-.025em b}\kern-.08em
    T\kern-.1667em\lower.7ex\hbox{E}\kern-.125emX}}

% Don't print section numbers
\setcounter{secnumdepth}{0}


\setlength{\parindent}{0pt}
\setlength{\parskip}{0pt plus 0.5ex}

\begin{document}
\raggedright
\footnotesize
\begin{multicols}{4}


% multicol parameters
% These lengths are set only within the two main columns
%\setlength{\columnseprule}{0.25pt}
\setlength{\premulticols}{1pt}
\setlength{\postmulticols}{1pt}
\setlength{\multicolsep}{1pt}
\setlength{\columnsep}{2pt}

\begin{center}
     \Large{\underline{\Title}} \\
\end{center}

\section{Equivalent Circuits}

If {\bf no dependent} sources, turn all of the current sources into open circuits and all of the voltage sources into short circuits, and solve for the equivalent resistance ($R_{Th} = R_{eq}$).

For any circuit, deactivate independent sources, add an external source $v_{ex}$, and solve for $i_{ex}$. $R_{Th} = v_{ex}/i_{ex}$.

For any circuit, find $v_{Th} = v_{oc}$ by solving the circuit with an open circuit at the terminals. Find $R_{Th}$ by using the fact that

$i_{sc} = \frac {V_{Th}} {R_{Th}}$

\section{Units}
\[
  Volt = V = \frac {kg \cdot m^2} {A \cdot s^3} = A \cdot \Omega = \frac {W} {A} = \frac {J} {C}
\]
\[
  Ampere = A = \frac {C} {s}
\]
\[
  Coulomb = C = F \cdot V = A \cdot s
\]

\[
Farad = F
= \frac{A \cdot s}{V}
= \frac{J}{V^2}
= \frac{W \cdot s}{V^2}
= \frac{C}{V}
\]
\[
F
= \frac{C^2}{J}
= \frac{C^2}{N \cdot m}
= \frac{s^2 \cdot C^2}{m^2 \cdot kg}
= \frac{s^4 \cdot A^2}{m^2 \cdot kg}
= \frac{s}{\Omega}
\]

\section{Circuit Properties}
\[
  M(\omega_c) = \frac {M_0} {\sqrt{2}}
\]
\[
  B = \begin{cases}
    0 \le \omega < \omega_c & \text{lowpass} \\
    \omega > \omega_c & \text{highpass} \\
    \omega_{c1} < \omega < \omega_{c2} & \text{bandpass} \\
    \omega < \omega_{c1} \text{ and } \omega > \omega_{c2} & \text{bandreject} \\
  \end{cases}
\]
\subsection{Quality Factor Q}
  \
\[
  Q = 2 \pi \left.\left( \frac {W_{stor}} {W_{diss}} \right) \right|_{w=w_0}
\]
\subsection{Impedance Matching}
Maximum Power Transfer occurs when \[ Z_L = Z_S^* \]
Minimum reflection occurs when \[ Z_L = Z_S \]

\columnbreak
\section{Op-Amps}
\begin{itemize*}
  \item Infinite voltage gain ($G \rightarrow \infty$).
  \item Infinite input impedance ($R_{in} \rightarrow \infty$).
  \begin{itemize*}
    \item Implies $i_p = i_n = 0$.
  \end{itemize*}
  \item Zero output impedance ($R_{out} \rightarrow 0$).
  \item For negative feedback configurations only, $v_p = v_n$.
  \item Analyze normally, but not at output (voltage not well-defined there).
\end{itemize*}
%\section{Transient Analysis}
\section{Resistors}
\begin{itemize*}
  \item $i,v$ relation $i = v/R$
  \item $v,i$ relation $v = iR$
  \item Power $p = i^2R$
  \item Stored energy $w = 0$
  \item Series combination $R_{eq} = R_1 + R_2$
  \item Parallel combination $R_{eq} = \frac {R_1R_2} {R_1 + R_2}$
  \item Allows instantaneous $v$ and $i$ change
\end{itemize*}
\section{Capacitors}
\begin{itemize*}
  \item $i,v$ relation $i_C = C\frac {dv_C} {dt}$
  \item $v,i$ relation $v(t) = v(t_0) + \frac {1} {C} \int_{t_0}^{t} i dt$
  \item Initially acts like {\bf short} circuit.
  \item $q = Cv$
  \item Power $p = Cv \frac {dv} {dt}$
  \item Stored energy $w = (1/2)Cv^2$ (should be in Joules)
  \item Combine {\bf opposite} of resistors
  \item Series combination $C_{eq} = \frac {C_1C_2} {C_1 + C_2}$
  \item Parallel combination $C_{eq} = C_1 + C_2$
  \item Becomes {\bf open} if left in DC for a long time
  \item {\bf Voltage} across can't change instantly ($v_0 = v_{0-}$).
  \item Allows instantaneous $i$ change
  \item Natural response $v(t) = [v(t_0) - v(\infty)] e^{-(t - t_0) / \tau)}$
  \item $\tau = \frac {1} {RC}$
\end{itemize*}
%\includegraphics[scale=0.15]{Discharging_capacitor.png}
\section{Inductors}
\begin{itemize*}
  \item $i,v$ relation $i_L =  i(t_0) + \frac {1} {L} \int_{t_0}^{t} v dt$
  \item $v,i$ relation $v_L = L \frac {di_L} {dt}$
  \item Initially acts like {\bf open} circuit.
  \item Becomes {\bf shorted} if left in DC for a long time
  \item $p = Li\frac {di} {dt}$
  \item Stored energy $W = \frac {1} {2} L i^2$ (should be in Joules).
  \item Combine like resistors
  \item Series combination $L_{eq} = L_1 + L_2$
  \item Parallel combination $L_{eq} = \frac {L_1L_2} {L_1 + L_2}$
  \item {\bf Current} across can't change instantly ($i_0 = i_{0-}$).
  \item Allows instantaneous $v$ change
  \item Natural response $i(t) = i(\infty) + [i(t_0) - i(\infty)](e^{-(t-t_0) / \tau})$
  \item $\tau = \frac {1} {RL}$
\end{itemize*}

\section{RLC Circuits}
Diff e.q. $x'' + ax' + bx = c$

Initially $x(0)$ and $x'(0)$

Finally $x(\infty) = \frac {c} {b}$,
$\alpha = \frac {a} {2}$,
$\omega_0 = \sqrt{b}$.

$ \omega_c = \omega_0 = \frac {1} {\sqrt{LC}} $,
$ \omega_d = \sqrt{w_0^2 - \alpha^2} $

\subsection{Series}
$ x(t) = v(t) $,
$ \alpha = \frac {R} {LC} $

\subsection{Parallel}
$ x(t) = i(t) $,
$ \alpha = \frac {1} {2RC} $

\subsection{Overdamped Response $\alpha > \omega_0$}
\[ x(t) = [x(\infty) + A_1e^{s_1t} + A_2e^{s_2t}]u(t) \]
\[ s_1 = -\alpha + \sqrt{\alpha^2 - \omega_0^2} \]
\[ s_2 = -\alpha - \sqrt{\alpha^2 - \omega_0^2} \]
\[ A_1 = \frac {x'(0) - s_2[x(0) - x(\infty)]} {s_1 - s_2} \]
\[ A_2 = -\frac {x'(0) = s_1[x(0) - x(\infty)]} {s_1 - s_2} \]
\subsection{Critically Damped $\alpha = \omega_0$}
\[ x(t) = [x(\infty) + (B_1 + B_2t)e^{-\alpha t}]u(t) \]
\[ B_1 = x(0) - x(\infty) \]
\[ B_2 = x'(0) + \alpha[x(0) - x(\infty)] \]
\subsection{Underdamped $\alpha < \omega_0$}
\[ x(t) = x(\infty) + [D_1cos\omega_dt + D_2sin\omega_dt]e^{-\alpha t}u(t) \]
\[ D_1 = x(0) - x(\infty) \]
\[ D_2 = \frac {x'(0) + \alpha[x(0) - x(\infty)]} {\omega_d} \]
\[ \omega_d = \sqrt{\omega_0^2 - \alpha^2} \]

\subsection{Bandpass}
$ B = \omega_{c2} - \omega_{c1} = \frac {R} {L} $
$ \omega_0 = \sqrt{\omega_{c1}\omega_{c2}} $
$ Q = \frac {\omega_0} {B} $
\[
  \omega_{c1} = - \frac {R} {2L} + \sqrt{\left(\frac {R} {2L}\right)^2 + \frac 
  {1} {LC}}
\]
\[
  \omega_{c2} = \frac {R} {2L} + \sqrt{\left(\frac {R} {2L}\right)^2 + \frac 
  {1} {LC}}
\]
\subsection{Series Bandpass}
$ B = \frac {R} {L} $
$ Q = \frac {\omega_0L} {R} $
\subsection{Parallel Bandpass}
$ B = \frac {1} {RC} $
$ Q = \frac {R} {\omega_0L} $

\section{Dividers}
Current divider: 
\[
  i_1 = \frac {R_2} {R_1 + R_2}i_{source}
\]
Voltage divider:
\[
  V_{mid} = \frac {R_{ground}} {R_{ground} + R_{source}}
\]

\section{Nodal Analysis}
Identify all extraordinary nodes, and choose one to be ground. Use KCL at each other node, with currents {\bf leaving} the node. All coefficients of the voltage at the node in the resulting equation should be positive (i.e. $V_1 / R$, not $-V_1/R$).

\subsection{Supernode}
If a voltage source connects two nodes, create one KCL equation for both. Also use $V_p - V_n = V_s$.

\section{Mesh Analysis}
Assign a clockwise current for each loop. Apply KVL and solve the resulting equations.

\subsection{Supermesh}
If a current source is between two meshes, create one KVL equation for both. Also use $i_1 - i_2 = i_{1,s}$.

\section{$Y \rightarrow \Delta$ Transformation}

$R_a$ between $2, 3$. $R_b$ between $1, 3$. $R_c$ between $1, 2$.

\[
  R_a = \frac {R_1R_2 + R_2R_3 + R_1R_3} {R_1}
\]

\[
  R_b = \frac {R_1R_2 + R_2R_3 + R_1R_3} {R_2}
\]

\[
  R_c = \frac {R_1R_2 + R_2R_3 + R_1R_3} {R_3}
\]

\section{$\Delta \rightarrow Y$ Transformation}

\[
  R_1 = \frac {R_bR_c} {R_a + R_b + R_c}
\]
\[
  R_2 = \frac {R_aR_c} {R_a + R_b + R_c}
\]
\[
  R_3 = \frac {R_aR_b} {R_a + R_b + R_c}
\]

\section{Wheatstone Bridge}
\[
  V_{out} \approx \frac {V_0} {4} \left( \frac {\Delta R} {R} \right)
\]

\section{Superposition}
If there are only independent sources, they can be solved in disjoint sets, with all other's disabled.
\end{multicols}


\newpage

\begin{multicols}{2}

\includegraphics[scale=0.35]{BasicOp-AmpConfigurations.png}

\section{Source Transform}
\includegraphics[scale=0.4]{Sourcetrans.jpg}

$V = IZ$, $I = \frac {V} {Z}$

\columnbreak

\begin{multicols}{2}

\section{Instrumentation Amplifier}
\includegraphics[scale=0.4]{400px-Op-Amp_Instrumentation_Amplifier.png}

$Gain = \frac {V_{out}} {V_2 - V_1} = \left(1 + \frac {2R_1} {R_{gain}}\right) \frac {R_3} {R_2}$

\section{Inductance Gyrator}
\includegraphics[scale=0.4]{300px-Op-Amp_Gyrator.png}

\section{Negative Impedance Converter}
\includegraphics[scale=0.4]{300px-Op-Amp_Negative_Impedance_Converter.png}

$R_{in} = -R_3 \frac {R_1} {R_2}$

\section{Non-Ideal Op-Amp}
\includegraphics[scale=0.4]{400px-Op-Amp_Internal.png}

\section{Sallen-Key}
\includegraphics[scale=0.4]{400px-Sallen-Key_Generic_Circuit.png}
\[
  \frac {v_{out}} {v_{in}} = \frac {Z_3Z_4} {Z_1Z_2 + Z_3(Z_1 + Z_2) + Z_3Z_4}
\]
\[
  v_x = v_{out} \left( \frac {Z_2} {Z_4} + 1 \right)
\]
\end{multicols}
\end{multicols}
\end{document}
